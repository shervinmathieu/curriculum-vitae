\cvsection{Expérience}

\begin{cventries}
	\cventry
	{Data Ingénieur Freelance / Betclic}
	{Extia}
	{Bordeaux, France}
	{septembre 2024 - présent}
	{
		\begin{cvitems}
			\item {Prestation en tant que Data Ingénieur Freelance pour Betclic Group.}
			\item {Participatgion à la migration des pipelines Big Data de dbt Cloud à dbt Core + Airflow.}
			\item {Développement de nouvelles pipelines pour les flux de paiement et de détection de fraude et blanchiment d'argent.}
			\item {Stack dbt + Amazon Web Services / Snowflake.}
		\end{cvitems}
	}

	\cventry
	{Data Ingénieur Freelance / Sanofi}
	{Eviden}
	{Pessac, France}
	{septembre 2023 - septembre 2024}
	{
		\begin{cvitems}
			\item {Prestation en tant que Data Ingénieur Freelance pour Sanofi.}
			\item {Maintenance du produit data de suivi de la généalogie des batchs, en Spark Scala et GraphQL.}
			\item {Migration des pipelines et datalake vers dbt et Snowflake.}
			\item {Stack Amazon Web Services / Snowflake.}
		\end{cvitems}
	}
	
	\cventry
	{Lead Data Ingénieur}
	{DataFret}
	{Bordeaux, France}
	{avril 2023 - septembre 2023}
	{
		\begin{cvitems}
			\item {Transformation de l'application SAS DataFret d'une version locale à une version cloud.}
			\item {Développement de nouvelles fonctionnalités ainsi que mise en place de l'architecture cloud.}
		\end{cvitems}
	}
	
	\cventry
	{Consultant Tech Lead Data Ingénieur / Volvo Trucks}
	{Thélio}
	{Bègles, France}
	{août 2022 - mars 2023}
	{
		\begin{cvitems}
			\item Lead Databricks pour deux équipes composées de data ingénieurs et data analysts, pour un projet de reporting de KPI.
			\item Suivi et correction des incidents techniques, échanges et partage de connaissances avec l'équipe.
			\item Mise en place d'une nouvelle architecture pour simplifier le travail de l'équipe ainsi que résoudre les problèmes techniques existants.
			\item Environnement international, communication en anglais.
			\item Stack Microsoft Azure.
		\end{cvitems}
	}
	
	\cventry
	{Consultant Data Ingénieur / Volvo Trucks}{}{}
	{février 2022 - juillet 2022}
	{
		\begin{cvitems}
			\item Mise en place d'ETL ainsi que d'une API REST pour fournir des données de maintenance prédictive à une application de type dashboard, destinée aux clients de Volvo Trucks.
			\item Amélioration de plus de 10x de la performance de l'API existante après modification de l'architecture data.
			\item Environnement international, communication en anglais.
			\item Stack Microsoft Azure.
		\end{cvitems}
	}
	
	\cventry
	{Consultant Data Ingénieur / La Banque Postale}{}{}
	{avril 2021 - janvier 2022}
	{
		\begin{cvitems}
			\item Intégration de la squad data, avec prise en main de multitude de tâches différentes, de la maintenance à la création de nouveaux flux ETL.
			\item Stack Cloudera on-premise.
		\end{cvitems}
	}
	
	\newpage

	\cventry
	{Consultant Data Ingénieur / Valemo}{}{}
	{février 2021 - avril 2021}
	{
		\begin{cvitems}
			\item Participation à la mise en place d'ETL divers avec Databricks.
			\item Stack Microsoft Azure.
		\end{cvitems}
	}
	
	\cventry
	{Consultant Data Ingénieur / 3w Régie (CDiscount Advertising)}{}{}
	{février 2021 - mars 2021}
	{
		\begin{cvitems}
			\item 
			\item Parsing et cleaning des données de marketing pour stockage dans BigQuery.
			\item Scrapping d'API fournisseur pour mise à jour des données journalières dans BigQuery.
			\item Stack Google Cloud Platform avec données en temps réel.
		\end{cvitems}
	}
	
	\cventry
	{Ingénieur d'études et développement}
	{LaBRI, Université de Bordeaux}
	{Talence, France}
	{octobre 2018 - février 2021}
	{
		\begin{cvitems}
			\item Participation au projet de structuration de recherche SysNum.
			\item Data Ingénieur en équipe avec un administrateur système.
			\item Mise en place d'un pipeline data pour la collecte d'informations des capteurs de température et présence au sein de l'université.
			\item Développement d'un réseau de neurones pour la recherche dans la prédiction de séries de temps.
			\item Développenent en Spark Java/Scala, Python, Typescript avec une stack on-premise.
		\end{cvitems}
	}
	
\end{cventries}